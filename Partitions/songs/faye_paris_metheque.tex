%!TEX root=../partitions.tex
[F-] ~ [Eb] ~ [Bb-] ~ [F-]
J'ai débarqué Paris d'un monde où l'on te rêve
J'ai fui les périls, les déserts où l'on crève
Tu m'as ouvert tes bras, toi ma Vénus de Milo
Tu brillais trop pour moi, je n'ai vu que ton halo
C'est pour ça que je l'ouvre, ma gueule est un musée
Je vis loin du feutré et des lumières tamisées
Dans tes ruelles cruelles ou tes boulevards à flics
Dans la musique truelle des silences chaophoniques
[F-] ~ [G#] ~ [Eb] ~ [Bb-] ~ [F-]
Paris ma belle beauté, tes prétendants se bousculent
Dans le brouillard épais de tes fines particules
Moi pour te mériter, je t'écrirai des poèmes
Que je chanterai la nuit tombée debout sur la scène


Paris s'éveille sous un ciel océanique
L'accent titi se mêle à l'Asie, l'Amérique, l'Afrique
Je suis une fleur craintive dans les craquelures du béton
A gagner deux sous, à dormir dessous les ponts
Paris bohème, Paris métèque, Paris d'ancre et d'exil
"Je piaffe l'amour" médite une chinoise à Belleville
Leonardo da Vinci se casse le dos sur un chantier
Je vois la vie en rose dans ces bras pakistanais
Il tourne le gyrophare, petit cheval de carrousel
Galope après les tirailleurs qui rétrécissent la tour Eiffel
D'un squat, d'un bidonville, d'une chambre de bonne ou d'un foyer
Je t'écris des poèmes où des fois je veux me noyer


Une ville de liberté pour les différents hommes
Des valises d'exilés, des juifs errants et des roms
Aux mémoires de pogrom, aux grimoires raturés
Des chemins d'Erevan, aux sentiers de Crimée
Caravanes d'apatrides, boat people, caravelle
Sur tes frontons Paris viennent lire l'universel
Et souvent je t'en veux, dédaigneuse et hautaine
Capitale de la monde a joué la mondaine
Laisse-nous consteller la vraie nuit que tu ignores
Cesse donc de faire briller les milles feux de ton décor
Paris ma belle je t'aime quand la lumière s'éteint
On écrit pas de poèmes pour une ville qui en est un
Paris ma belle je t'aime quand la lumière s'éteint
On écrit pas de poèmes pour une ville qui en est un
Paris ma belle je t'aime quand la lumière s'éteint
On écrit pas de poèmes pour une ville qui en est un