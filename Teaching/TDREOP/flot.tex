\documentclass{beamer}

\usepackage{config_latex/mypackages}
\usepackage{config_latex/mypackages_beamer}
%\input{config_latex/mynotations.tex}

\newcommand{\val}{\mathrm{val}}

\title{Problèmes de flots}
\author[Maël Forcier]{Maël Forcier}
\institute[REOP 2020]{REOP 2020}
\date{14/10/2020}

\setbeamertemplate{navigation symbols}{}

\begin{document}

\begin{frame}
 \maketitle
\end{frame}



\begin{frame}{Plan}
 \tableofcontents
\end{frame}

\section{Flot maximal et coupe minimale}

\subsection{Définitions}

\begin{frame}[t]{Cadre} 
Soit $D = (V, A)$ un graphe orienté avec : \pause
\begin{itemize} 
  \item sur chaque arc $a \in A$ une capacité ~$u(a) \geq 0$ \pause
  \item deux noeuds spéciaux: une source~$s$ et un puit ~$t$ (pour target en anglais)
\end{itemize}
\end{frame}

\begin{frame}[t]{Définition d'un ~$s-t$ flot} \pause
  Un~$s-t$ flot est une fonction~$f: A \longmapsto \Rr_+$ qui satisfait la loi de Kirchhoff \pause
  \begin{align*}
    \forall v \in V \backslash \{s, t\}, \quad \sum_{a \in \delta^-(v)} f(a) = \sum_{a \in \delta^+(v)} f(a)
  \end{align*} \pause
  et les contraintes de capacité \pause
  \begin{align*}
    \forall a \in A, \quad f(a) \leq u(a)
  \end{align*}
\end{frame}

\begin{frame}[t]{Valeur d'un~$s-t$ flot}
  La valeur d'un ~$s-t$ flow~$f$ est "la quantité qui sort de la source" : \pause
  \begin{align*}
    \val(f) = \sum_{a \in \delta^+(s)} f(a) - \sum_{a \in \delta^-(s)} f(a) 
  \end{align*}
  \pause
  Exercice: Calculer un ~$s-t$ flot maximal dans le graphe suivant :
    \begin{figure}
    \centering
    \includegraphics[scale=0.45]{images/flow_example.png}
\end{figure}
\end{frame}


\begin{frame}[t]{Définition d'une coupe} 
  Une ~$s-t$ coupe ~$(S, T)$ est une partition de l'ensemble des sommets $V = S \sqcup T$ telle que ~$s \in S$ et ~$t \in T$.

  \pause
  
  
  La capacité d'une coupe est la somme des capacités des arcs qui la traversent : \pause
  \begin{align*}
    u(S, T) = \sum_{\substack{i \in S,~ j \in T\\(i,j) \in A}} u(i,j)
  \end{align*}

   On peut aussi définir la coupe comme un ensemble d'arcs qui ~$B = \delta^+(S)$ intersectent tous les $s-t$ chemins.
   
  \begin{align*}
  	u(B) = \sum_{a \in B} u(a)
  \end{align*}
\end{frame}

\begin{frame}{Flot maximal et coupe minimale} \pause
  \vfill
  \begin{exampleblock}{Problème du flot maximal}
    Trouver un ~$s-t$ flot de valeur maximale ~$\val(f)$ 
  \end{exampleblock}
  \vfill \pause
  \begin{exampleblock}{Problème de la coupe minimale}
    Trouve une coupe ~$s-t$ de capacité minimale ~$u(B)$
  \end{exampleblock}
  \vfill
\end{frame}

% \begin{frame}{Exercises} \pause
%   \begin{enumerate}
%     \item Battle on a network (6.18)
    
%     \textit{\tiny A command center is located at a vertex~$p$ of a non-directed network. We know the position of subordinates modeled by a subset~$S$ of the vertices of the network. We want to destroy a minimum number of links in order to prevent any communication between the command center and the subordinates. How to solve this problem in polynomial time?}
%   \end{enumerate}
% \end{frame}



\begin{frame}[t]{Coupe et flot : majoration}
  \begin{proposition}[6.3]
    Soit ~$f \leq u$ un ~$s-t$ flot et ~$B$ une ~$s-t$ coupe.\\
    Alors ~$\val(f) \leq u(B)$.
  \end{proposition}
  \pause
Exercice: Calculer un ~$s-t$ flot maximal dans le graphe suivant :
    \begin{figure}
    \centering
    \includegraphics[scale=0.45]{images/flow_example.png}
\end{figure}

\end{frame}

\subsection{Graphe résiduel et Ford Fulkerson}
\begin{frame}[t]{Graphe résiduel}
  Pour tout arc~$a = (i, j) \in A$, on définit l'arc~$\overleftarrow{a} = (j, i)$  \pause et les capacités résiduelles :
  \begin{align*}
    u_f(a) = u(a) - f(a) +f\left(\overleftarrow{a}\right)
  \end{align*}

  \vfill \pause
  
  \begin{figure}
    \centering
    \includegraphics[width=0.8\linewidth]{images/residual_graph_1.png}
    \caption{Les arcs en gras ont des capacités résiduelles non nulles.}
  \end{figure}
\end{frame}

\begin{frame}[t]{Graphe résiduel}
  Le graphe résiduel est le graphe ~$D_f = (V, A_f)$ avec
  \begin{align*}
    A_f = \{a \in A \cup \overleftarrow{A}:~ u_f(a) > 0\} 
  \end{align*}
  munis des capacités $u_f$
  
  \vfill \pause
  
  \begin{figure}
    \centering
    \includegraphics[width=0.8\linewidth]{images/residual_graph_2.png}
    \caption{Le graphe résiduel $D_f$ deduit de $D$}
  \end{figure}
\end{frame}

\begin{frame}[t]{Chemin dans le graphe résiduel}
  Un ~$s-t$ chemin dans le graphe résiduel ~$D_f$ permet de trouver un flot de plus grande valeur.
  
  \vfill \pause
  
  \begin{figure}
    \centering
    \includegraphics[width=0.8\linewidth]{images/residual_graph_3.png}
  \end{figure}
\end{frame}

\begin{frame}[t]{Critère d'optimalité} \pause
  \begin{theorem}[6.4]
    Un~$s-t$ flot ~$f$ est maximal ssi il n'existe pas de chemin dans son graphe résiduel ~$D_f$. 
  \end{theorem}
\end{frame}

\begin{frame}{Illustration du critère d'optimalité} \pause
  \begin{figure}
    \begin{minipage}{0.65\linewidth}
      \includegraphics[width=\linewidth]{images/residual_graph_4.png}
    \end{minipage}
    \begin{minipage}{0.3\linewidth}
      \caption{Un $s-t$ flot maximal}
    \end{minipage}
  \end{figure}
  \vfill \pause
  \begin{figure}
    \begin{minipage}{0.65\linewidth}
      \includegraphics[width=\linewidth]{images/residual_graph_5.png}
    \end{minipage}
    \begin{minipage}{0.3\linewidth}
      \caption{Une ~$s-t$ coupe de capacité $0$ dans le graphe résiduel.}
    \end{minipage}
  \end{figure}
\end{frame}

\begin{frame}[t]{Théorème Max flow / min cut} \pause
  \begin{theorem}[6.5]
    La valeur maximale d'un $s-t$ flot est égale à la capacité minimale d'une $s-t$ coupe. 
  \end{theorem}
\end{frame}


\begin{frame}[t]{Ford-Fulkerson}
  \setcounter{algocf}{4}
  \vfill
  \begin{algorithm}[H]
        \caption{Algorithme de Ford-Fulkerson}
   \SetAlgoLined 
   ~$f(a):= 0$ pour tout ~$a \in A$\; \pause
   \While{il existe un chemin dans le graphe résiduel}{ \pause
    Sélectionner $P$ un $s-t$ chemin dans le graphe résiduel\;
    Augmenter ~$f$  le long de ~$P$ par ~$\min_{a \in P} u_f(a)$\;
  } \pause
   Retourner ~$f$
  \end{algorithm}
  \vfill 
\end{frame}

% \begin{frame}[t]{Complexity of Ford-Fulkerson (1)} \pause
%   \begin{proposition}
%     Each iteration of the Ford-Fulkerson loop takes~$O(|A|)$ time
%   \end{proposition}
%   \vfill \pause
%   \begin{proposition}
%     If the capacities $u$ are integral, so are the flow augmentations.
%   \end{proposition}
%   \vfill
% \end{frame}

% \begin{frame}[t]{Complexity of Ford-Fulkerson (2)} \pause
%   \begin{theorem}
%      If the capacities $u$ are integral, the Ford-Fulkerson algorithm returns a maximum~$s-t$ flow in~$O(|A| \times \val_{\max})$ time, where~$\val_{\max}$ is the maximum value of an~$s-t$ flow.
%   \end{theorem}
% \end{frame}

% \begin{frame}[t]{Edmonds-Karp}
%   \setcounter{algocf}{4}
%   \begin{algorithm}[H]
%         \caption{Edmonds-Karp algorithm}
%    \SetAlgoLined \pause
%    \KwData{a digraph~$D = (V, A)$ with capacities~$u$, two vertices~$s$ and~$t$}
%    \KwResult{an~$s-t$ flow of maximum value} \pause
%    Set~$f(a) = 0$ for all~$a \in A$\; \pause
%    \While{there is an~$f$-augmenting path}{ \pause
%     Select an~$f$-augmenting path~$P$ \textcolor{red}{with minimum number of edges}\; 
%     Augment~$f$ along~$P$ by~$\min_{a \in P} u_f(a)$\;
%   } \pause
%    Return~$f$
%   \end{algorithm}
%   \vfill \pause
%   Questions: \pause
%   \begin{itemize}
%     \item How do we select such an augmenting path? \pause
%     \item Why does it improve the complexity?
%   \end{itemize}
% \end{frame}

% \begin{frame}{Complexity of Edmonds-Karp} \pause
%   \begin{proposition}
%     The Edmonds-Karp loop is crossed at most~$|A| \times |V|$ times.
%   \end{proposition}  \pause
%    We can show that
%     \begin{itemize}
%       \item The (unweighted) distance~$\mathrm{dist}_{D_f}(s, t)$ in the residual graph is nonincreasing
%       \item It can only remain constant for at most~$|A|$ iterations
%     \end{itemize}
%     \pause
%   \begin{theorem}
%      The Edmonds-Karp algorithm returns a maximum~$s-t$ flow in~$O(|A|^2 \times |V|)$ time.
%   \end{theorem}
% \end{frame}

% \begin{frame}
% \only<1>{
% \frametitle{Modélisation du réseau ferroviaire soviétique}
%   \begin{figure}
%     \includegraphics[height=0.6\linewidth]{images/ussr_railway_network.png}
%   \end{figure}
%   }
% \only<2>{
% \frametitle{Trouver une coupe}
%   \begin{figure}
%     \includegraphics[height=0.6\linewidth]{images/ussr_max_flow.png}
%   \end{figure}
%   }
% \end{frame}


\section{Programmation linéaire pour les flots}

\begin{frame}{Programmation linéaire pour les flots maximaux} \pause
  On peut reformuler le problème de flot maximal comme de la programmation linéaire : \pause
  \begin{align*}
    \max \quad & \sum_{a \in \delta^+(s)} x_a - \sum_{a \in \delta^-(s)} x_a \\
    \mathrm{s.t.} \quad &
      \sum_{a \in \delta^-(v)} x_a = \sum_{a \in \delta^+(v)} x_a \quad \quad \forall~ v \in V \backslash \{s, t\} \\
      & 0 \leq x_a \leq u(a) \quad \quad \forall~ a \in A
  \end{align*}
\end{frame}

\begin{frame}[t]{Propriété des LP de type max flow} \pause
  \begin{proposition}[6.11]
    La matrice des contraintes d'un LP de type max flow est totalement unimodulaire.
  \end{proposition}
  
  \vfill \pause

  \begin{proposition}[6.12]
    Le problème de coupe minimale est le dual du problème de flot maximal.
  \end{proposition}
  
  \vfill
\end{frame}





\section{$b$-flot à coût minimal}

\subsection{Définitions}

\begin{frame}[t]{Cadre}  \pause
Soit ~$D = (V, A)$ un graphe orienté avec : \pause
\begin{itemize}
  \item des bornes inférieures et supérieurs sur les capacités ~$0 \leq \ell(a) \leq u(a)$ pour chaque arc \pause
  \item des coûts ~$c(a) \geq 0$ pour chaque arc $a \in A$ \pause
  \item une "source" algébrique ~$b(v) \in \Rr$ à chaque sommet
\end{itemize}

\end{frame}

\begin{frame}[t]{Définition d'un $b$-flot} \pause
  Un ~$b$-flot est une fonction ~$f: A \longmapsto \Rr_+$ satisfaisant la loi de  Kirchhoff
  \begin{align*}
    \forall v \in V, \quad b(v) + \sum_{a \in \delta^-(v)} f(a) = \sum_{a \in \delta^+(v)} f(a)
  \end{align*} \pause
  et les contraintes de capacité
  \begin{align*}
    \forall a \in A, \quad \ell(a) \leq f(a) \leq u(a)
  \end{align*} \pause
  Si ~$b(v) = 0$ partout, on dit que ~$f$ est une circulation.
  
  \medskip \pause
  
  Le coût d'un ~$b$-flot~$f$ est
  \begin{align*}
    c(f) = \sum_{a \in A} c(a) f(a)
  \end{align*}

  On doit avoir 
$\sum_{v \in V} b(v) = 0$ si on veut satisfaire la loi de Kirchhoff.
\end{frame}



\begin{frame}[t]{Graphe résiduel et cycles} \pause
  Pour tout arc~$a=(i, j) \in A$, on définit l'arc~$\overleftarrow{a} = (j, i)$ et les capacités résiduelles :
  \begin{align*}
    u_f(i,j) = u(i,j) - f(i,j)+f(j,i) -l(j,i) 
  \end{align*} \pause
  On étend ~$c$ en définissant ~$\tilde c(i,j) = c(i,j)-c(j,i)$.

\medskip \pause

  Le graphe résiduel est le graphe ~$D_f = (V, A_f)$, munis des capacités $u_f$ et des coût $\tilde c$ avec
  \begin{align*}
    A_f = \{a \in A \cup \overleftarrow{A}:~ u_f(a) > 0\} 
  \end{align*} \pause
  
  On peut construire un algorithme en regardant les cycles dans le graphe résiduel.
  On définit le coût d'un cycle par
  \begin{align*}
    c(C) = \sum_{a \in C} c(a)
  \end{align*}
\end{frame}

% \begin{frame}[t]{Optimality criterion} \pause
%   \begin{theorem}[6.9]
%     A~$b$-flow~$f$ is minimal iff there is no~$f$-augmenting cycle with negative cost.
%   \end{theorem} \pause
%   \begin{proof}
%     \only<3>{
%       Suppose~$C$ is an~$f$-augmenting cycle with negative cost. We can augment the flow along~$C$ by~$\min_{a \in C} u_f(a)$ to get~$f'$, and see that:
%         \begin{itemize}
%           \item~$f'$ is still a~$b$-flow because 
%           \begin{itemize}
%             \item Kirchhoff's current law is preserved since~$C$ is a cycle
%             \item capacity constraints are still satisfied by choice of~$\min_{a \in C} u_f(a)$
%           \end{itemize}
%           \item~$c(f') = c(f) + \underbrace{c(C)}_{<0} \times \underbrace{\min_{a \in C} u_f(a)}_{>0} < c(f)$ 
%         \end{itemize}
%         Therefore~$f$ does not have a minimum cost.
%     }
%     \only<4>{
%     Now suppose~$f$ is not a minimum cost~$b$-flow, and let~$g$ be another~$b$-flow of smaller cost.
%     \begin{enumerate}
%       \item The difference~$g-f$ is a circulation on~$D_f$
%       \begin{itemize}
%         \item~$g-f$ satisfies Kirchhoff's current law with~$b = 0$
%         \item~$\forall~a \in A, \quad 0 \leq g(a) - f(a) \leq u(a) - f(a) = u_f(a)$
%       \end{itemize}
%       \item Any circulation~$h$ is a positive linear combination of cycles.
%       \begin{itemize}
%         \item This is proved by recursion on the number of nonzero arcs in~$h$
%       \end{itemize}
%       \item By 1 and 2,~$g-f$ can be decomposed as a positive linear combination of cycles in~$D_f$, i.e.~$f$-augmenting cycles. Since~$c(g-f) = c(g) - c(f) < 0$, at least one of these cycles has negative cost.
%     \end{enumerate}
%     }
%   \end{proof}
% \end{frame}

% \subsection{Algorithm}

% \begin{frame}[t]{Cycle-canceling algorithm \citep{goldbergFindingMinimumcostCirculations1989}} \pause
%   \setcounter{algocf}{4}
%   \begin{algorithm}[H]
%         \caption{Cycle-canceling algorithm}
%    \SetAlgoLined \pause
%    \KwData{a digraph~$D = (V, A)$ with capacities~$l \leq u$, costs~$c$ and inputs~$b$}
%    \KwResult{a~$b$-flow of minimum cost}  \pause
%    Find an initial~$b$-flow~$f$\; \pause
%    \While{there is an~$f$-augmenting cycle with negative cost}{ \pause
%     Select an~$f$-augmenting cycle~$C$ \textcolor{red}{with minimum mean cost}\; 
%     Augment~$f$ along~$C$ by~$\min_{a \in C} u_f(a)$\;
%   } \pause
%    Return~$f$
%   \end{algorithm}
%   \vfill \pause
%   Questions: \pause
%   \begin{itemize}
%     \item How do we find an initial~$b$-flow? \pause
%     \item How do we select an~$f$-augmenting cycle with minimum mean cost?
%   \end{itemize}
% \end{frame}

% \begin{frame}[t]{Complexity of cycle-canceling (1)} \pause
%   \begin{proposition}[ex 6.4]
%     An initial~$b$-flow can be found in~$O(|A| \times |V|)$ time.
%   \end{proposition}
%   \vfill \pause
%   \begin{proposition}[ex 6.3]
%     An $f$-augmenting cycle of minimum mean cost can be found in~$O(|A| \times |V|)$ time.
%   \end{proposition}
%   \vfill
% \end{frame}

% \begin{frame}{Complexity of cycle-canceling (2)} \pause
%   \begin{theorem}[6.10]
%     The cycle-canceling algorithm returns a minimum cost~$b$-flow in~$O(|A|^3 |V|^2 \log|V|)$ time.
%   \end{theorem}
% \end{frame}



\begin{frame}{Exercices}
    \only<1>{Monge's transportation problem (ex. 6.9)\\
    
    Consider~$m$ holes that we want to fill using~$n$ piles of sand. Let us call~$s_i$ the mass of the~$i$-th pile of sand and~$t_j$ the mass of sand necessary to fill the~$j$-th hole. For each couple~$(i,j)$ we know the distance~$d_{ij}$ of the~$i$-th pile to the~$j$-th hole. If a mass~$x_{ij}$ is moved from pile~$i$ to hole~$j$, the cost of the displacement is equal to~$d_{ij}x_{ij}$. We want to find the transportation plan that allows the holes to be filled at the minimum cost.
    }
    \only<2>{
    Seat alocation of a bus company (ex. 6.10)\\
    
    A bus capable of carrying not more than B passengers will depart from city 1 and visit the cities $2,3,\cdots,n$ successively.
    The number of passengers wanting to travel from city $i$ to city $j$ (with $i<j$) is $d_{i,j}$ ande the price of this trip is $p_{i,j}$. How many passengers should we take in each city to maximize total revenue ? Model this problem as a flow problem.
    }
    \only<3>{
    Taxi fleet (ex. 6.13)\\
    
    A cab company has~$p$ passenger journeys to complete over a day. For each of these journeys~$i=1,\ldots,p$, they know its departure location~$o_i$ and its departure time~$h_i$, as well as the duration of the trip~$t_i$ and its arrival location~$d_i$. Moreover, the time~$\tau_{ji}$ to get from~$d_j$ to~$o_i$ is known for all couples~$(i,j)$. The company wants to minimize the number of cabs needed to satisfy the demand. All cabs are assumed to be located in one depot at the beginning of the day.
    }
\end{frame}


\end{document}