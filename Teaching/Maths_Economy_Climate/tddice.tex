
\documentclass[12pt]{article}
%\documentclass[smallextended,numbook]{svjour3}

\usepackage[english]{babel}
\usepackage[utf8x]{inputenc}
\usepackage{pdfpages}
\usepackage{amsfonts,dsfont}
\usepackage{amsmath}
\usepackage{amssymb}
\usepackage{amsthm} %commenter pour style MP
\usepackage{bm}
\usepackage{natbib}
\usepackage{hyperref}
\usepackage{algorithm2e}
\usepackage{enumitem}
\usepackage[capitalise]{cleveref}
%\usepackage[capitalise,poorman]{cleveref}

\setlength {\marginparwidth }{2cm}
\usepackage{todonotes}
\usepackage[all]{xy} 

\makeatletter \let\cl@part\relax \makeatother

%\usepackage[top=1cm, bottom=1cm, left=1cm, right=1cm]{geometry}
\usepackage[top=2cm, bottom=2cm, left=2cm, right=2cm]{geometry}
\usepackage{xcolor}
\usepackage{tikz}
\usetikzlibrary{calc}
\usetikzlibrary{intersections}
\tikzset{offset/.style={to path={%
    -- ($(\tikztostart)!#1cm!(\tikztotarget)$)}},
         offset/.default=1}
\tikzset{>=latex}


\renewcommand{\thesection}{\Alph{section}}
\usepackage{pgfplots}

\newcommand{\mf}[1]{\begin{color}{brown}\texttt{MF:#1}\end{color}}

\newcommand{\TODO}{\begin{color}{red}TODO\end{color}}

\newcommand{\tco}{tCO_2eq}

\usepackage{subcaption}
%\usepackage[colorinlistoftodos,bordercolor=orange,backgroundcolor=orange!20,linecolor=orange,textsize=scriptsize]{todonotes}


%% Theorems
\newtheorem{hypo}{Assumption}
\Crefname{hypo}{Assumption}{Assumptions}

\newtheorem{theorem}{Theorem}
\newtheorem{lemma}[theorem]{Lemma}
\newtheorem{remark}[theorem]{Remark}
\newtheorem{cor}[theorem]{Corollary}
\newtheorem{prop}[theorem]{Proposition}
\newtheorem{defi}[theorem]{Definition}
\newtheorem{conj}{Conjecture}
\newtheorem{nota}{Notation}
\theoremstyle{remark}
\newtheorem{exa}{Example}





\title{TD \\Modèle 
DICE (Dynamic Integrated Climate Economy) \\
de Nordhaus simplifié }

\author{Maël Forcier}


\begin{document}
\maketitle


\section{Modélisation}

Le modèle de Nordhaus est un modèle de macroéconomie qui étudie l'évolution de l'économie mondiale. Le modèle est dynamique, les variables non-constantes seront indicés par le temps $t$. La variable principale est le capital, noté $K_t$, c'est-à-dire la valeur en \$ de tous les biens matériels ou immatériels dans le monde. Le produit intérieur brut (PIB) en \$ noté $Q_t$ est la somme de tous les revenus annuels d'une économie. La consommation notée $C_t$ est la somme en \$ de tous les biens et services perissables utilisés pendant une année.L'investissement en \$ noté $I_t$ est l'ensemble des 

\begin{enumerate}
\item On suppose que le PIB n'est utilisé que pour la consommation et l'investissement. Proposer une équation reliant $Q_t$, $C_t$ et $I_t$.
\end{enumerate}
\textit{Réponse :} $Q_{t}=C_t + I_t$
\begin{enumerate}[resume]
\item  Le capital accumulé se déprécie à un taux $\delta_K$ qui le fait diminuer entre chaque étape, mais l'investissement permet de générer du nouveau capital. Proposer une équation dite de dynamique reliant $K_{t}$, $K_{t-1}$, $I_t$ et $\delta_K$.
\end{enumerate}
\textit{Réponse :} $K_{t}=(1-\delta_K)K_{t-1}+I_t$


Variable émissions de GES en $G\tco$\\
Variable PIB en Md\$\\
Variable Température \\
Variable Population \\
Contrôle taxe carbone en $\$ \backslash \tco$ \\
Part de l'économie carbonée en \% \\
Variable intensité carbone du PIB en $\tco \backslash \$ $
\\
Dynamique : \\
Rétroaction négative de la hausse de la température sur le $PIB$ \\
Taxe carbone fait ralentir la part d'économie carbonée à court terme\\
Taxe carbone fait augmenter l'investissement en décarboné à long terme.\\
Selon Eurostat, l'investissement représente 22 \% du PIB contre 78 \% pour la consommation. Pour simplifier, on fixe à $I_t=Q_t/4$

Les équations de température

\section{Scénario contrôle total}


\section{Scénario sans contrôle}


\section{Discussion et critiques}
\begin{enumerate}[resume]
\item Quelles variables sont endogènes, exogènes ?
\item Comment qualifier le modèle top-down/bottom-up, statique/dynamique, stochastique/déterministe, d'optimisation, discret/continu ?
\item Quelles hypothèses pourraient être ajoutées ?
\end{enumerate}

\section{Comparaison avec le papier d'origine}
\begin{enumerate}[resume]
\item Quelles sont les simplifications que l'on a faite par rapport au modèle DICE du papier de Nordhaus de 1992 ?
\end{enumerate}


\end{document}