%!TEX root=../partitions.tex

[A-] ~ [G] ~ [F] ~ [G] ~ [A-]
J'ai rencontré ce matin devant la haie de mon champ
Une troupe de marins d'ouvriers de paysans
Où allez-vous camarades avec vos fusils chargés
Nous tendrons des embuscades viens rejoindre notre armée


La voilà la Blanche Hermine vive la mouette et l'ajonc
La voilà la Blanche Hermine vive Fougères et Clisson!

Où allez-vous camarades avec vos fusils chargés
Nous tendrons des embuscades viens rejoindre notre armée
Ma mie dit que c'est folie d'aller faire la guerre aux Francs
Mais je dis que c'est folie d'être enchaîné plus longtemps

La voilà la Blanche Hermine vive la mouette et l'ajonc
La voilà la Blanche Hermine vive Fougères et Clisson!


Elle me dit que c'est folie d'aller faire la guerre aux Francs
Mais je dis que c'est folie d'être enchaîné plus longtemps
Elle aura bien de la peine pour élever les enfants
Elle aura bien de la peine car je m'en vais pour longtemps

La voilà la Blanche Hermine vive la mouette et l'ajonc
La voilà la Blanche Hermine vive Fougères et Clisson!

Elle aura bien de la peine pour élever les enfants
Elle aura bien de la peine car je m'en vais pour longtemps

Je viendrai à la nuit noire tant que la guerre durera
Comme les femmes en noir triste et seule elle m'attendra

La voilà la Blanche Hermine vive la mouette et l'ajonc
La voilà la Blanche Hermine vive Fougères et Clisson!

Je viendrai à la nuit noire tant que la guerre durera
Comme les femmes en noir triste et seule elle m'attendra
Et sans doute pense-t-elle que je suis en déraison
De la voir mon coeur se serre là-bas devant la maison


La voilà la Blanche Hermine vive la mouette et l'ajonc
La voilà la Blanche Hermine vive Fougères et Clisson!

Et sans doute pense-t-elle que je suis en déraison
De la voir mon coeur se serre là-bas devant la maison
Et si je meurs à la guerre pourra-t-elle me pardonner
D'avoir préféré ma terre à l'amour qu'elle me donnait

La voilà la Blanche Hermine vive la mouette et l'ajonc
La voilà la Blanche Hermine vive Fougères et Clisson!


Et si je meurs à la guerre pourra-t-elle me pardonner
D'avoir préféré ma terre à l'amour qu'elle me donnait
J'ai rencontré ce matin devant la haie de mon champ
Une troupe de marins, d'ouvriers, de paysans

La voilà la Blanche Hermine vive la mouette et l'ajonc
La voilà la Blanche Hermine vive Fougères et Clisson! 