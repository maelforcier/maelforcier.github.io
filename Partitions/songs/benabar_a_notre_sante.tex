%!TEX root=../partitions.tex
\begin{guitar}
[G] ~ [B-] ~ [C] ~ [D7]
Ils mènent une vie sans excès.
Y'a pas de doute ils se surveillent de près.
Avoir un corps parfait, c'est un sacerdoce.
Mais leur capital santè mérite des sacrifices.
Il boit de la biere sans alcool.
Elle mange pas de viande ça donne du cholesterol.
ils boivent leur café décaféiné.
Avec du sucre dé-sucrifié.

[E-] ~ [B-7] ~ [CM7] ~ [D7]
Est-ce de ma faute à moi, si j'aime le café et l'odeur du tabac,
Me coucher tard la nuit, me lever tôt l'après midi,
Aller au resto et boire des apéros
... A notre santé !

Elle met de la crème anti-âge.
Qu'elle combine avec un doux gommage.
Qui restructure en profondeur les macromolécules
En hydratant le derme contre les rides et les ridules
Comme elle redoute l'effet peau d'orange,
Elle a eu un rameur pour leurs dix ans de mariage .

Il dit qu'il aime le sport… pas la compètition,
C'est quoi ces coupes, ces rameurs, bien en vue dans le salon ?

Est-ce de ma faute à moi, si j'aime le café et l'odeur du tabac,
Me coucher tard la nuit, me lever tôt l'après midi,
Aller au resto et boire des apéros
... A notre santé !

Les cheveux blancs des vieux, Les enfants dépeignés,
Les rides au coin des yeux, les doigts dans le nez
Le désordre, le bordel et le bruit, le pas bien rangé…

Le " ça peut plus durer ! "
.. A notre santé.
Parfois un criminel allume une cigarette,
Elle le fusille du regard et court vers la fenêtre.
Elle dit " Ha ! De l'air, c'est vivifiant ! ".
Et aspire à plein poumons les bon gaz d'echappement
Il a des bombes qui vaporisent du poison…
Contre tous les insectes de la création.
Il faut éradiquer tout c'qui apporte des maladies.
Il a des doutes sur ses voisins. Mais les tuer, c'est interdit.

Est-ce de ma faute à moi, si j'aime le café et l'odeur du tabac,
Me coucher tard la nuit, me lever tôt l'après midi,

Aller au resto et boire des apéros
... A notre santé ! 

[G] ~ [B-] ~ [A-] ~ [D7]
\end{guitar}