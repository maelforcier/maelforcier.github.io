%!TEX root=../partitions.tex

\begin{guitar}

  [C]  On la trou[Dm7]vait plutôt jo[C]lie, Li[G7]ly,
 [C]  Elle arri[Dm7]vait des Soma[C]lies, Li[G]ly,
 [G]  Dans un ba[F]teau plein d'émi[Em]grés
Qui vena[Dm]ient tous de leur plein  [C]gré
Vider les  [G7] poubelles à Pa[C]ris.
 
 
 [G7] ~  [C]  ~ [G7] 
 
 
 Elle croyait qu'on était égaux, Lily,
 Au pays d'Voltaire et d'Hugo, Lily,
 Mais pour Debussy en revanche
 Il faut deux noires pour une blanche
 Ça fait un sacré distinguo.

 [C] Elle aimait [F]tant la liber[G7]té, Li[C]ly,
 [E7] Elle rêvait de fraterni[Am]té, Lily,
 [Am]Un hôte [G7] lier rue Secré[C]tan
Lui a pré[G7]cisé en arri[C]vant
Qu'on ne re[B7]cevait que des [E] blancs.  [E7] 

Elle a déchargé des cageots, Lily,
Elle s'est tapé les sales boulots, Lily,
Elle crie pour vendre des choux-fleurs,
Dans la rue, ses frères de couleur
L'accompagnent au marteau-piqueur.
 
 
Et quand on l'appelait Blanche-Neige, Lily,
Elle se laissait plus prendre au piège, Lily,
Elle trouvait ça très amusant,
Même s'il fallait serrer les dents,
Ils auraient été trop contents.
Elle aima un beau blond frisé, Lily,
Qui était tout prêt à l'épouser, Lily,
Mais la belle-famille lui dit :" nous
Ne sommes pas racistes pour deux sous
Mais on ne veut pas de ça chez nous".
Elle a essayé l'Amérique, Lily,
Ce grand pays démocratique, Lily,
Elle aurait pas cru sans le voir
Que la couleur du désespoir
Là-bas aussi ce fût le noir.
 
 
Mais, dans un meeting à Memphis, Lily,
Elle a vu Angela Davis, Lily,
Qui lui dit :" viens ma petite soeur
En s'unissant, on a moins peur
Des loups qui guettent le trappeur".
Et c'est pour conjurer sa peur, Lily,
Qu'elle lève aussi un poing rageur, Lily,
Au milieu de tous ces gugusses,
Qui foutent le feu aux autobus
Interdits aux gens de couleur.
Mais, dans ton combat quotidien, Lily,
Tu connaîtras un type bien, Lily,
Et l'enfant qui naîtra un jour
Aura la couleur de l'amour
Contre laquelle on ne peut rien.
 
 [G7]   [C]   [G7] 
 
On la trouvait plutôt jolie, Lily,
Elle arrivait des Somalies, Lily,
Dans un bateau plein d'émigrés
Qui venaient tous de leur plein gré
Vider les poubelles à Paris.

\end{guitar}